\documentclass[a4paper,9pt]{extarticle}

\usepackage[utf8]{inputenc}
\usepackage{geometry}
\geometry{letterpaper, margin=0.75in}
\usepackage{titlesec}
\usepackage{enumitem}
\usepackage{hyperref}
\usepackage{tabularx}
\usepackage{fancyhdr}

\setlist{noitemsep}
\titleformat{\section}{\large\bfseries}{}{1em}{}[\titlerule]
\titlespacing*{\section}{0pt}{\baselineskip}{\baselineskip}

\pagestyle{fancy}
\renewcommand{\headrulewidth}{0pt}
\fancyhead{}
\fancyhead[L]{\textit{Y\'emalin Morel KPAVODE}}
\fancyhead[R]{\textit{March 2025}}
\thispagestyle{empty}

\begin{document}

\begin{flushleft}
\textbf{\LARGE Y\'emalin Morel KPAVODE}\\[2pt]
Junior Data Scientist \textendash{} Passionn\'e d\textquoteright{}IA appliqu\'ee au d\'eveloppement, Abomey-Calavi, B\'enin\\
\href{mailto:yemalem03@gmail.com}{yemalem03@gmail.com} | \href{tel:+22995754157}{+229 95 75 41 57} | \href{https://www.linkedin.com/in/morel-kpavode}{linkedin.com/in/morel-kpavode} | \href{https://github.com/yemal22}{github.com/yemal22}
\end{flushleft}

\section*{R\'ESUM\'E}
\noindent
\textit{\small\textbf{\textsf{\`{E}tudiant d\'etermin\'e avec une double formation en math\'ematiques fondamentales et technologies de l'information, je me sp\'ecialise en intelligence artificielle et Big Data avec pour objectif de concevoir des solutions technologiques adapt\'ees aux d\'efis socio-\'economiques du B\'enin. Mes projets s\textquoteright{}inscrivent dans les Objectifs de D\'eveloppement Durable, notamment dans la sant\'e publique, la s\'ecurit\'e alimentaire et l\'innovation num\'erique.}}}

\section*{DOMAINES DE RECHERCHE}
Computer Vision, Deep Learning, Big Data, Intelligence Artificielle appliqu\'ee au d\'eveloppement, NLP, IA pour l'Afrique, l'environnement et la sant\'e publique.

\section*{COMP\'ETENCES}
\begin{itemize}
    \item \textbf{IA et Vision par ordinateur :} TensorFlow, PyTorch, OpenCV, IA G\'en\'erative, CNN, Transformer, Hugging Face
    \item \textbf{Science des donn\'ees :} Python, Scikit-learn, Pandas, SQL, Matplotlib, Seaborn, R
    \item \textbf{D\'eveloppement Web & DevOps :} Django, Flask, REST API, Docker, Kubernetes, Git, Linux
    \item \textbf{Méthodologie :} TDD, Méthodes Agiles (Scrum, Kanban)
\end{itemize}

\section*{LANGUES}
\begin{itemize}
    \item Fran\c{c}ais : Natif
    \item Anglais : Interm\'ediaire (niveau B1)
\end{itemize}

\section*{\'EDUCATION}


\noindent
\textbf{Institut de Math\'ematiques et des Sciences Physiques (IMSP)}, Dangbo, B\'enin \hfill Oct 2023 -- Ao\^ut 2024\\
Licence professionnelle en Technologies de l\textquoteright{}Information et de la Communication


\noindent
\textbf{Facult\'e des Sciences et Techniques (FAST)}, Abomey-Calavi, B\'enin \hfill Oct 2020 -- Sept 2023\\
Licence en Math\'ematiques Fondamentales

\section*{EXP\'ERIENCE PROFESSIONNELLE}
\textbf{Analyse de donn\'ees statistiques -- LESCAL} \hfill Abomey-Calavi, B\'enin\\
\textit{Prestataire externe (Freelance)} \hfill Jan 2025 -- F\'ev 2025
\begin{itemize}
    \item Traitement et analyse de donn\'ees statistiques selon les exigences du cahier des charges du laboratoire.
    \item Mission r\'ealis\'ee avec rigueur, valid\'ee par attestation officielle.
    \item \textbf{Technologies} : Python, Pandas, Statistiques appliqu\'ees, Visualisation.
\end{itemize}

\section*{EXP\'ERIENCE ACAD\'EMIQUE}
\textbf{D\'eveloppeur Back-end -- Plateforme Pehu (Architecture microservices)} \hfill B\'enin\\
\textit{Juil 2024 -- D\'ec 2024}
\begin{itemize}
    \item Conception et d\'eveloppement d\textquoteright{}APIs REST en JavaScript pour modules internes.
    \item Mise en \oe uvre de TDD (Test-Driven Development) pour garantir la fiabilit\'e du code.
    \item Int\'egration de composants en langage C dans le backend.
    \item Contribution \`a l\textquoteright{}internationalisation multilingue de la plateforme.
    \item \textbf{Technologies} : C, JavaScript, Mithril.js, REST API, TDD.
\end{itemize}

\section*{PROJETS DE RECHERCHE}
\textbf{Syst\`eme d\textquoteright{}annotation automatique d\textquoteright{}images africaines} \hfill B\'enin\\
\textit{Projet personnel} \hfill Mars 2025 -- Pr\'esent
\begin{itemize}
    \item D\'eveloppement d\textquoteright{}un pipeline pour l\textquoteright{}annotation d\textquoteright{}images de mode vestimentaire et de plats africains.
    \item Fusion de datasets vari\'es pour cr\'eer une base enrichie \`a des fins de classification et reconnaissance d\textquoteright{}objets.
    \item \textbf{Objectif} : promouvoir la vision par ordinateur appliqu\'ee \`a la valorisation des cultures africaines.
    \item \textbf{Technologies} : Python, Huggingface, FastAPI, YOLOv5, Pandas.
\end{itemize}

\textbf{G\'en\'eration automatique de descriptions d\textquoteright{}images} \hfill B\'enin\\
\textit{Projet de startup} \hfill F\'ev 2025 -- Pr\'esent
\begin{itemize}
    \item Conception d\textquoteright{}un mod\`ele de g\'en\'eration de textes pour d\'ecrire automatiquement des images de meubles africains.
    \item \textbf{Technologies} : TensorFlow, CNN, Transformer, OpenCV.
\end{itemize}

\textbf{Syst\`eme de recherche intelligente} \hfill B\'enin\\
\textit{Projet personnel} \hfill Jan 2025 -- Pr\'esent
\begin{itemize}
    \item Moteur de recherche multimodal (texte/image/voix) avec indexation automatique.
    \item \textbf{Technologies} : Python, NLP, IA G\'en\'erative, Traitement vocal.
\end{itemize}

\section*{CERTIFICATIONS}
\textbf{IBM Professional Certificate in Data Science} \hfill Coursera \hfill F\'ev 2025\\
\textbf{Certificat en Intelligence Artificielle} \hfill Universit\'e Virtuelle du S\'en\'egal \hfill En cours


\end{document}
