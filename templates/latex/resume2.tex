\documentclass[a4paper,10pt]{article}
\usepackage[utf8]{inputenc}
\usepackage[T1]{fontenc}
\usepackage[margin=1in]{geometry}
\usepackage{titlesec}
\usepackage{enumitem}
\usepackage{hyperref}
\usepackage{xcolor}
\usepackage{graphicx}
\usepackage{fancyhdr}

\pagestyle{fancy}
\fancyhf{}
\renewcommand{\headrulewidth}{0pt}
\fancyhead[L]{\textit{Yémalin Morel KPAVODE}}
\fancyhead[R]{\textit{CV – {{ now|date:"F Y" }} }}

% Style titres de sections
\titleformat{\section}
  {\large\bfseries\color{black}}
  {}
  {0pt}
  {\thesection.}
  [\titlerule]

\titlespacing{\section}{0pt}{1.2em}{0.5em}
\setlist{nosep,leftmargin=*}

\definecolor{graytext}{gray}{0.35}
\color{graytext}
\renewcommand{\familydefault}{\sfdefault}

\begin{document}

% ==== EN-TÊTE ====
\begin{center}
  {\Huge \textbf{Yémalin Morel KPAVODE}}\\
  \vspace{2pt}
  Junior Data Scientist – IA pour l’Afrique, Bénin\\
  \href{mailto:yemalem03@gmail.com}{yemalem03@gmail.com} • 
  \href{tel:+22995754157}{+229 95 75 41 57} • 
  \href{https://linkedin.com/in/morel-kpavode}{linkedin.com/in/morel-kpavode} • 
  \href{https://github.com/yemal22}{github.com/yemal22}
\end{center}

\vspace{0.5em}
\hrule
\vspace{1.2em}

% ==== RÉSUMÉ ====
\section*{Résumé}
\noindent
Étudiant déterminé avec une double formation en mathématiques fondamentales et TIC, je me spécialise en intelligence artificielle et Big Data pour proposer des solutions aux défis socio-économiques de l'Afrique. Mes projets s’inscrivent dans les Objectifs de Développement Durable : santé, alimentation, culture numérique.

% ==== DOMAINES DE RECHERCHE ====
\section*{Domaines de recherche}
Computer Vision, Deep Learning, IA générative, Big Data, NLP, IA pour les cultures africaines

% ==== COMPÉTENCES ====
\section*{Compétences}
\begin{itemize}
  \item \textbf{IA / Vision :} TensorFlow, PyTorch, Hugging Face, OpenCV
  \item \textbf{Data Science :} Python, Scikit-learn, Pandas, SQL, Matplotlib
  \item \textbf{Dev Web / DevOps :} Django, Flask, REST API, Docker, Git
  \item \textbf{Méthodologie :} TDD, Méthodes agiles (Scrum, Kanban)
\end{itemize}

% ==== LANGUES ====
\section*{Langues}
\begin{itemize}
  \item Français : natif
  \item Anglais : intermédiaire (niveau B1)
\end{itemize}

% ==== ÉDUCATION ====
\section*{Éducation}

\textbf{{ edu.institution }}, {{ edu.location }} \hfill {{ edu.start_date|date:"Y" }} -- {{ edu.end_date|date:"Y" }}\\
{{ edu.degree }}\\
\vspace{0.4em}


% ==== EXPÉRIENCE PROFESSIONNELLE ====
\section*{Expérience professionnelle}

\textbf{{ exp.title }} – {{ exp.company }} \hfill {{ exp.start_date|date:"F Y" }} – {{ exp.end_date|date:"F Y" }}\\
\textit{{ exp.location }}\\
\begin{itemize}
  \item {{ exp.description|striptags|linebreaksbr|safe }}
\end{itemize}
\vspace{0.5em}


% ==== EXPÉRIENCE ACADÉMIQUE ====
\section*{Expérience académique}

\textbf{{ exp.title }} – {{ exp.company }} \hfill {{ exp.start_date|date:"F Y" }} – {{ exp.end_date|date:"F Y" }}\\
\textit{{ exp.location }}\\
\begin{itemize}
  \item {{ exp.description|striptags|linebreaksbr|safe }}
\end{itemize}
\vspace{0.5em}


% ==== PROJETS ====
\section*{Projets}

\textbf{{ p.title }} \hfill {{ p.start_date|date:"F Y" }} – {{ p.end_date|date:"F Y" }}Présent\\
\textit{{ p.context }} – {{ p.location }}\\
\begin{itemize}
  \item {{ p.description|striptags|linebreaksbr|safe }}
  
  \item \textbf{Technologies :} {{ p.technologies.all|join:", " }}
  
\end{itemize}
\vspace{0.5em}


% ==== CERTIFICATIONS ====
\section*{Certifications}

\textbf{{ c.name }} – {{ c.provider }} \hfill {{ c.date_issued|date:"F Y" }}\\


\end{document}
